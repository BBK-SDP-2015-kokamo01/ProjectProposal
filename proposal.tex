\documentclass[a4paper, 11pt]{article}
%\usepackage[parfill]{parskip}

%\title{Some Title I haven't decided yet}
%\author{Keimi Okamoto}
\usepackage[T1]{fontenc}

\newlength{\drop}

%%%%%%%START%%%%%%%
\begin{document}
  \begin{titlepage}
    \drop=0.1\textheight
    \centering
    \vspace*{\baselineskip}
    \rule{\textwidth}{1.6pt}\vspace*{-\baselineskip}\vspace*{2pt}
    \rule{\textwidth}{0.4pt}\\[\baselineskip]
    {\Large{MSc Computer Science\\[0.3\baselineskip] }}
    {\huge{Project Proposal\\[0.3\baselineskip] }}
	
    \rule{\textwidth}{0.4pt}\vspace*{-\baselineskip}\vspace{3.2pt}
    \rule{\textwidth}{1.6pt}
    \\[\baselineskip]
    \scshape
    {\Large Some Title That I haven?t Decided Yet \\}
    Location, date from--to\par
    \vspace*{2\baselineskip}
    %Edited by \\[\baselineskip]
    {\normalsize\emph{Supervisor: }{\large Professor George Roussos\par}}
    {\normalsize\emph{Author: }{\large Keimi Okamoto\par}}
    
    {\itshape 2015}
    \vfill
    {\large BIRKBECK UNIVERSITY OF LONDON\par}
{\footnotesize DEPARTMENT OF COMPUTER SCIENCE \& INFORMATION SYSTEMS}\par
  \end{titlepage}
  
%\maketitle
%........................
% Contents
%........................
\tableofcontents
\clearpage

%........................
% Abstract
%........................
\abstract{Abstract}
blah blah
\clearpage

%........................
% Introduction
%........................
\section{Introduction}

\subsection{Background}
Over production of produce is a global issue with food waste exceeding two billion tonnes annually. Such large quantities of waste has sever negative social, environmental and economical impacts. The multifaceted nature of the food supply chain gives ample opportunity for waste to occur and inadequate waste prevention methods could cause figures to rise. Further more, such complex pipelines poses great difficulties in the accurate quantification of waste generated, meaning that the figures are likely to be higher than reported. A clear understanding of the chain is necessary to identify the vulnerable components where waste can arise, below is a brief description of each of the three main entities that comprise of the chain and their accountability to the statistics.

\paragraph{Producers}At the start exists the agriculturalists and farmers who cultivate plants and livestock. These entities serve as the primary source of supply to the market and respond to orders placed by retailers. Over production is often encouraged by the merchant to compensate for the possibility of an unfruitful harvest or an unexpected rise in demand. 

\paragraph{Retailers}The intermediator between producer and consumer are the retailers and vendors, the most dominant being Supermarkets, namely, Sainsbury?s, Tesco, Asda and Morrison. Perpetual competition for marketshare leads to aggressive advertising and competitive pricing wars, with millions of pounds at stake urgency for quantity control becomes less of an priority. Errors in sales forecasts can result to wastage or idle stock occupying valuable real-estate. Having more stock than necessary is unfavourable for business as the rental of warehouses can cost millions annually. Vendors Ascetically unpleasing but perfectly consumable food is is deemed unworthy of stocking and rejected.

\paragraph{Consumers}At the end of the chain are the consumers that are regularly influenced by enticing ?buy-one-get-one-free? offers and other multi-buy promotions. Retailers markdown items that are nearing expiration to sell  to the consumer as an attempt to compensate for potential losses. This method of damage control while beneficial to the supermarkets have negative implications on the consumers. Loss can also occur due to basic human errors of simply forgetting to consume the food in time. Fast paced lifestyles coupled with unpredictability attribute to the difficulties in keeping track of past purchases and expiration dates, contributing to overflowing landfills. 

\vspace{\baselineskip}

This project will take a consumer-centric approach to tackle the issue of waste. The United Kingdom alone is estimated to generate 15 million tonnes of food waste every year, 7 million of which is accountable to domestic households. Having stated this, retailers contribute largely to the figures as they strive to meet contrived demands, self-orchestrated by marketing to boost revenue, causing waste that would have resided at the retailer to be pushed down the chain and ultimately dumped with the consumers. By providing consumers tools to better manage their inventory that would aid a lifestyle of maximal resource utilisation with minimum waste, demand could be controlled keeping waste at bay with the retailers. Essentially creating an upstream ripple effect and discouraging the over production of food by targeting the root cause of the problem.



\vspace{\baselineskip}
\vspace{\baselineskip}
\vspace{\baselineskip}

\subsection{Problem}
\paragraph{Environmental} When food is wasted, this is the direct repercussion of over production and a needless contribution to the expanding carbon foot print. Processes such as pesticide application, cooking, packaging creation and disposal, distribution and temperature controlled storage all require copious amounts of fuel and energy. Waste is being generated at such a rapid rate maintaining this in landfill sites is becoming unfeasible.

\paragraph{Economical} Typically a UK household has been reported to throw away an average of \pounds940 worth of food annually. Amounting to roughly 50kg of waste that must be collected, managed and recycled, putting pressure on councils, all of which can result in higher taxes and wasted resources.

\paragraph{Social} Influenced by the retailers and succumbing to the bargain deals, customers frequently over purchase food causing over consumption. Needlessly consuming to avoid loss can pose serious health risks such as obesity, diabetes, high blood pressure, and cardiovascular diseases, these are potentially life threatening and can impact the populations life expectancy and add pressure on the health care system. 

\paragraph{Human memory} Failed stock keeping of household items is one of the primary causes of food expiring before consumption. The human brain has limited capacity to store and recall information. Research by George Armitage Miller, a prominent figure in the field of cognitive phycology discovered that the number of objects an average human can hold in working memory is seven, give or take two.[2] Foods vary in categories such as meats, fish, fruit, vegetables and nearly all come with different expiration dates, relying on memory alone is impractical. 

\paragraph{LifeStyle} Busy schedules dissuade people to use produces brought in advance and instead opt for the quick and easier choice of eating out, thus items purchased with the intentions of consumption end up as waste. Combining compatible ingredient to create an appetising dish, in addition the complication of prioritising use-by-dates of fresh produces can be time-consuming and arduous task. Households usually have multiple residents and double purchasing of items is common due to lack of communication. 

\vspace{\baselineskip}

**Image of pie goes here** // put in Appendix
Image will explain-Supermarkets and other retailers contribute almost 2 million tonnes to the statistics and Approximately 3.6 million tonnes of the waste comes from the manufactures such as farmers and the agricultural industry. 

\vspace{\baselineskip}
\vspace{\baselineskip}
\vspace{\baselineskip}

\subsection{Current Measures to Combat Waste}

\paragraph{Governments} Various campaigns have been launched with the purpose to educate consumers on the implications of food waste and waste prevention methods. Such organisations as Waste \& Resources Action Programme (WRAP), a registered charity part funded by the UK Government, have been raising awareness by interacting with communities and working to promote waste avoidance. A campaign launched by WARP ?Love Food, Hate Waste? (LFHW), which primarily operates through an interactive website advocates recycling and provides the viewer with helpful tips. Utilisation of social media advertising of Twitter and Facebook are integral to the success of their operation together with the distribution of a mobile phone application with features such as shopping lists, recipe suggestions, portion size suggestion in accordance to the number of people serving. Other governments are also promoting the use of mobile applications to combat waste, ?Smart Cooking? developed by the Netherlands Nutrition Centre Foundation (NNCF) funded by the Dutch government incorporates similar features to LFHW.

\paragraph{Anaerobic Digestion}Closed-loop solutions that convert bio-methane gas extracted from food waste into electricity have been introduced by Sainsbury?s and the waste management service Biffa. The primary goal being the powering of supermarkets with energy sourced form surplus and reducing strain on the landfill sites.

\paragraph{Wageningen University}Dutch researchers form NXP Semiconductors are working with the Netherlands Packaging Centre (NPC) a packaging solutions company, have developed a sensor enabled RFID tag developed to monitor environmental changes in the supply chain. The Pasteur sensor tag has the capacity of measuring environmental conditions such shifts in temperature and gas conditions registered during transportation and various stages of storage. This data is then calculated to give an accurate reading of the shelf life of a product. At the currents state this is only available at the producer-level and supplier-level for large crates of produces but item-level tagging is being considered for the future. 

\paragraph{Peking University}Scientists in Beijing have also developed a colour coded smart tag that uses nanotechnology to indicate when food is spoiling. The metallic nanorods in the gel mimic the length of time microbes propagate in foods, the more bacteria the further in the decomposition process it is. The tag can also react to varying tempters that can have an effect on the shelf life of a product, this could potentially remove the necessity of sell-by-dates. This is currently being meticulous tested to avoid any inaccuracy that could pose a potential health risk to consumers.
\paragraph{Smart Refrigerator}  This internet enabled appliance was designed for home food management including the automated replenishment of stock. The user will scan products using a reader and the fridge is able to keep track of foods and suggest recipes depending in accordance to the content.

\paragraph{TooSkee \& LeanPath} Food management mobile applications focusing on reducing waste and keep track of purchased items through a mobile application. Receipts are scanned and items are logged. Much like the other mobile applications it will suggest dishes and remind the user to consume products before the expiration date. LeanPath provides waste management solutions for the hospitality industry. Harvested data is analysed and losses and savings are reported to the users.

Physical interactions can be effective and inspirational but the labour force required to generate and sustain interest is costly and impractical, hence the movement towards digital mediums. Mobile application are an effective delivery method with most people owning smartphones, but in an over crowded market place where a single bad review can jeopardise the success of an App quality is paramount and users have become increasingly intolerant of a poor interface design or performance such as delayed content loading.

The eagerly anticipated smart fridge was somewhat anti-climatic as flaws in the practicality of the product surfaced. Items had to be manually entered due to the lack data and the recommendations were not as helpful as initially thought, together with the unit costing over \$20,000 many investors were reluctant to invest.


Cutting edge innovations such as the atmosphere and temperature sensing RFID tag and the nanotechnology tag are undeniably beneficial to the supply chain. The only bottleneck for the RFID  tag is the cost of production at the individual item level. Nanotechnology could potentially replace printed use-by-dates on products  and could provide users with the an accurate reading of the longevity of food but this requires the user to manually open the fridge and memorise the colour of the tags. Presently there is no method for the tags to communicate, the data generated could be valuable for waste management and even retailers. Transparency throughout the supply chain is paramount at this moment in time and is achievable with the aid of machine to machine communication. Products and prototypes mentioned here can be enhanced and with the necessary infrastructure to support it. This leads on to the next topic of the movement of the Internet Of Things.

\vspace{\baselineskip}
\vspace{\baselineskip}
\vspace{\baselineskip}


\subsection{Internet of Things and SmartCities}
\paragraph{A bit of history} The first appliance to go online was a Coca-Cola vending machine developed in Carnegie Mellon University in 1982. Users were able to connect via the internet and check if the canned beverages were chilled, this information would be the deciding factor on whether the user would make the trip to the machine. This technological advancement gave an insight into a new era were objects were able to cater to our immediate needs, depending on the current circumstance and deliver us information with which we are able to make an informed decision. With the enablement of machine to machine communications, a once passive and inanimate object is able to actively communicate with other ?things? through a network, sharing data and working harmoniously to maximise efficiency to permissively aid our everyday lives.

\paragraph{SmartCities} The motivation behind Smart Cities is the idea of a self sustaining ecosystem made up of active object, supporting the efficient and economical utilisation of resources. The digital regulation of sectors such as, mobility, home, energy and waste management assists a better quality of life for its citizens. Information and communications systems are designed to deal with autonomous fault detection and self-healing intelligence with minimal, if not, the complete elimination of human intervention. 

This notion of efficiency driven, interconnected networks can be applied to the supply chain. Interleaving processes can be machine managed, making it less error prone and complex pipelines and components become transparent. Allowing the promulgation of waste occurrence to it?s counterparts so that preemptive waste prevention measures could be established in advance. An example of this is if a store shelf holding boxes of cornflakes can report to the start of the chain that there are still x-units to sell, so for the next x dates cornflakes need not be produced. Giving opportunity for demand to catch up and reducing back-log of stock.

\paragraph{Product Identification: RFID vs Barcodes}Product identification and registration is fundamental for retailers to monitor stock and analyse sales totals. To accomplish this task, barcodes have been implemented in the supply chain since the 1970?s. This method of identification accelerate and digitalised the checkout process. 

For the barcode to be read successfully there must be no obstructions between the laser and the barcode, this includes dirt or scratches that distort the image. The laser must be kept parallel to the barcode for a successful read and simultaneous scanning is not possible. It is also note worthy to mention that barcodes are unique to the product type but not at the item-level. For example in a crate filled with milk made by the same manufacture will all have the same barcode thus a machine would not be able to distinguish one item from another. However they are universally recognised and inexpensive to print. 

Unlike barcodes lasers are not needed to for reads. RFID utilises electromagnetic radio fields for communication, so long long as the tag is within the vicinity of the field it is able be read or written and even allow multiple tags to be read simultaneously. Tag have varied memory capacity but typically very low. Usually the size of a URL can be written, this is enough to bridge between object and the web where additional data can be stored or updated as external data storage is abundant. URL?s also provide individuality to an object, allowing customised granular information to be stored and accessible at the item-level.


\vspace{\baselineskip}
\vspace{\baselineskip}
\vspace{\baselineskip}

\subsection{Importance of Item-level Identification}
\emph{'The amount of food waste in the industrialised countries exceeds the total first
production of the whole continent of Africa. This is an incredible waste of human effort
and environmental and economic cost. I say, ?On some estimates?, because we very
rapidly found that the estimates in this field are rather difficult, which limits the degree
to which the EU can play as effective a role as it perhaps ought. We found that
measurement of food waste at different stages of the chain and between different
countries was pretty incompatible. Until that is resolved, the EU level probably has to
be aspirational, exculpatory and a matter of learning from best practice. 12'}

\paragraph{Remote Amendment of Human Errors}
Foods is regularly recalled due to human errors in printing the incorrect information on the label or neglecting to provide information the doesn?t abide by regulations. Food perfectly fit for consupmtion is destined to become to waste by using RFID tags could remotely be updated, issuing immediate alerts to consumers of the mistake and the amended error. This provides a different approach to error management. With the information being online it can be accessible anywhere.

\paragraph{Food Safety and Traceability}
In the past there has been numerous incidents where products have been recalled due to the presence of bacteria or abnormalities in foods. A notable incident is the 2013 meat adulteration scandal in the EU where traces of horse meat where discovered in foods such as minced meat and ready prepared meals. The time and resources to trace trough the supply chain was estimated to have cost the Food Standard Agency (FSA) \pounds900,000 between 2011?2012. And a further \pounds1.6 million between 2012?13 \cite{3}. Other casualties include the reputations and integrity of the blameless producers falsely accused due to inaccurate data that implicated them as the guilty. 

With item-level identification the contaminated produce could be traced back immediately and the products recalled. For example if an infected animal is used in various products, all items holding that particular code can be instantly traceable, compartmentalising the outbreak and maximising efficiency in damage control. 

batches of meat may also be mixed. (In article 18 by the)
(*1)Environment, Food and Rural Affairs Committee: Evidence,
example:Article 3 of Commission Implementing Regulation EU

\paragraph{Transparency \& Consumer Rights}
The law enforces that labels of fresh meat must contain the country of origin, but this does not apply to the same meat that is processed, such as hamburgers, pies and sausages. (cite article 18) Meats may also be mixed providing that the animals are slaughtered in the same country, meaning a single hamburger could be made up of several cows. 

The Regulations of the EU parliament and council state that abattoirs and agriculturalists must provide documentation containing country of birth, rearing, slaughter, cutting and slaughterhouse and cutting plant approval numbers for the product(cite) when sourcing retailers or informing officials. But this is not enforced at the consumer level, meaning the information is not being passed down the supply chain. It is evident that meats have unique backgrounds, ranging from the rearing environments, type of feed consumed and drugs administered. This information can help make consumers make better decisions whether it is for the health, environmental or ethically conscious individuals. The Food Standard Agency published a report on the labelling guidance of food 

?It is clear that many consumers want more information on the origin of meat ingredients in meat products, and in the Agency?s consumer research the ingredients in dairy produce also score highly in this respect. The law requires an origin declaration on fresh beef but not on the same product when it has been seasoned. Providing information on the origin of all ingredients in all products would be disproportionately burdensome for industry, and would risk overloading the label with information that is not
seen as important by consumers.? - Food Standard Agency ?country of origin labelling guidence?

With the use of RFID the overloading of the label would no longer be a reason to with hold information from the consumer and the consumers as individuals can decide what information is of importance to them rather than a collective opinion that can be washed over.

Areas such as Japan where vegetables and livestock were exposed to radiation due to the nuclear leak[cite nuclear leak?] raises health concerns and articulate (transpicuous) detail of a product is high priority.


\paragraph{Supply Chain Intelligence} As well as waste management item-level identification can benefit other areas of the supply chain. Data analysis can be carried out to better business intelligence. Accurate behavioural analysis of consumers could minimise 
As with any data collection the issue of privacy arises but this is not within the scope of this project. 

Electronic product code (EPC)Developed in Massachusetts Institute of Technology Auto-ID Centre, with the aim of providing a unique identification for every physical object in the world. EPC Network

\clearpage
%........................
% Aims & Objectives
%........................
\section{Aims and Objectives}
\subsection{Aims}

The aim is to demonstrate the use of item-level RFID tagging and the positive contribution to waste reduction. Using the ubiquitous computing paradigm the application should seamlessly integrate into the domestic home environment. Enhancing the routinely activity of food supply replenishment and assisting the optimum utilisation of resources. 


\iffalse
 \paragraph{\textbf{\textit{Embedded}}}
 \begin{flushleft}  A fridge that contains a reader that reads tagged items A fridge that contains a reader that reads tagged items A fridge that contains a reader that reads tagged items A fridge that contains a reader that reads tagged items A fridge that contains a reader that reads tagged items.
  \end{flushleft}
  
   \paragraph{\textbf{\textit{Context Aware}}}
 \begin{flushleft}  A fridge that contains a reader that reads tagged items A fridge that contains a reader that reads tagged items A fridge that contains a reader that reads tagged items A fridge that contains a reader that reads tagged items A fridge that contains a reader that reads tagged items.
  \end{flushleft}
  
   \paragraph{\textbf{\textit{Personalised}}}
 \begin{flushleft}  A fridge that contains a reader that reads tagged items A fridge that contains a reader that reads tagged items A fridge that contains a reader that reads tagged items A fridge that contains a reader that reads tagged items A fridge that contains a reader that reads tagged items.
  \end{flushleft}
  
   \paragraph{\textbf{\textit{Adaptive}}}
 \begin{flushleft}  A fridge that contains a reader that reads tagged items A fridge that contains a reader that reads tagged items A fridge that contains a reader that reads tagged items A fridge that contains a reader that reads tagged items A fridge that contains a reader that reads tagged items.
  \end{flushleft}
  
     \paragraph{\textbf{\textit{Anticipatory}}}
 \begin{flushleft}  A fridge that contains a reader that reads tagged items A fridge that contains a reader that reads tagged items A fridge that contains a reader that reads tagged items A fridge that contains a reader that reads tagged items A fridge that contains a reader that reads tagged items.
  \end{flushleft}
\fi


\begin{itemize}
  %\setlength{\parskip}{0cm}%
  \item[] \textbf{Embedded}
  \begin{flushleft}  The fridge will be embedded with RFID readers and the groceries will be tagged holding an ID that corresponds to a URL where the item details will be accessible from any subscribing smart devices. The fridge will be responsible for the persistence of incoming data and the logging of any alteration to the state of the fridge. i.e If items are added or removed.
  \end{flushleft}
  \item[] \textbf{Context Aware} 
    \begin{flushleft} With the use of location sensors such as GPS, if one of the subscribing users is near a supermarket a notification containing an itemised list of products and the quality necessary to optimally restock relative to the number of inhabitants and the items previously purchased reducing impulse buying and over purchasing. (With the exception to override the feature if any of the members are expecting guests for diners.) Additionally, the contents of the fridge may also be observed through a smart device for quick reminders on what is back at home. At a particular timing, for example half an hour before arriving home from work, the application may recommend a quick and easy recipe inspiring the use of the produce purchased.
\end{flushleft}
  \item[] \textbf{Personalised} 
    \begin{flushleft} Automatic meal plan generation with the items in the fridge tailored to the dietary requirements of the diners. Personalised allergy warnings if a user attempts to remove something from the fridge that contains the offending item. An end of week analysis for the household of how much food was consumed in time and the translation to monetary savings as motivation.
\end{flushleft}
  \item[] \textbf{Adaptive} 
    \begin{flushleft} The system must adapt to the change in the environment. When more food is added to the fridge the prioritisation of consumption according to the expiration date must be kept ordered to minimise the possibility of waste. If one user alters the state of the fridge updates must be available to all other users informing of the change.
\end{flushleft}
  \item[] \textbf{Anticipatory} 
    \begin{flushleft} If the system sees that more food is in the fridge than what the family?s usual consumption value as sees a high possibility of waste tips to extend product life such as freezing or cooking and advice such as donating or sharing with the community or friends will be made. Sending out an alert to local charities or friends, neighbours to share the unwanted food. 
\end{flushleft}
\end{itemize}


\paragraph{Scenario}On the way home from work Rachel visits the supermarket. She consults her smartphone for a reminder of what her fridge contained and is a recommended shopping list has been generated for her. She will need less food than the previous week as she has a family dinner scheduled at her mother-in-laws this weekend. She browses the poultry aisle and notices a special offer on chicken, ?Buy two get the third free? the label reads. According to the application the fridge already contains chicken that must be consumed by tomorrow. She decides against the purchase and carries on, the next item on the list is milk. But then her phone notifies her that her husband Frank has just stocked the fridge with a one litre carton of semi-skimmed milk. She completes the shopping and arrives home and restocks the fridge. Her daughter Ingrid is on the way home form college, her parents are working late and she must prepare dinner for herself and younger brother Dean this evening. As she scrolls through the items on the screen of the smartPhone the app recommends cheese and onion quiche, ready in 20mins and one of Deans favourite. After they have had dinner Ingrid decides to prepare desert, as she unloads the cheesecake from the fridge her smartPhone signals an alert. The screen highlights a warning that the cheesecake contains gelatine. Ingrid is a vegetarian, she opts for the yoghurt instead and serves the cheesecake to her brother. At the end of the week a visual chart representing the analysis of the families savings and is broadcasted to all members.

\paragraph{Further Extension\dots} As with the internet of things as more objects join the internet the larger the network will grown and communication lines will open. Giving endless opportunity for the project to grow. Devices such as fitness trackers can connect and calculate the amount of calories consumed to the amount of calories burned. Applications such as MyGROCER will be able to detect when a product is 

\subsection{Objectives}

\begin{enumerate}
   \item \textbf{Embedded}
     \begin{flushleft}  A fridge that contains a reader that reads tagged items
   \end{flushleft}
   \item \textbf{Context Aware}
   \begin{flushleft} These devices can recognize family and family situational context
With the use of RFID it can recognise that a certain member of the family has approached the fridge and extracted an item. Dietary information such as calories, allergies can be displayed for a particular member in the house. 
  \end{flushleft}
   \item \textbf{Personalised}
   \begin{flushleft}they can be tailored to family needs
Prioritisation of fresh food consumption to minimise the possibility of waste. Automatic meal plan generation with the items in the fridge.
  \end{flushleft}
   \item \textbf{Adaptive}
   \begin{flushleft}they can change in response to the family
If food is nearing expiration a notification must be sent to subscribing users. 
In the context where there are multiple inhabitants the fridge must respond to updates from all subscribing users.  if one member stocks the fridge it must notify others deterring other members to also purchase.
  \end{flushleft}
   \item \textbf{Anticipatory}
   \begin{flushleft} they can anticipate family desires without conscious mediation.
Automatic shopping list generation in accordance to previous items stocked. Dietary requirements and the number of people i the house.  
 \end{flushleft}
\end{enumerate}

\subsection{Limitations}
\clearpage


%........................
% Development
%........................
\section{Development}
\subsection{Methodology}
\subsection{Framework}
\subsection{Architecture}
\subsection{Tools}

\clearpage

%........................
% Schedule
%........................
\section{Schedule}
\subsection{Timetable}

\clearpage

%........................
% Bibliography
%........................
\begin{thebibliography}{11}

\bibitem{code}
	fjdsfjds
	
\bibitem{2}
	G.A Miller, \emph{Magical Number Seven, Plus or Minus Two: Some Limits on Our Capacity for Processing Information}, vol. 63. Cambridge, MA: The Psychological Review, 1956.

\bibitem{3}
Environment, Food and Rural Affairs Committee: Evidence,
example:Article 3 of Commission Implementing Regulation EU


\end{thebibliography}

\end{document}

%%%
\setlength{\parindent}{0pt}
\begin{document}


\end{document}
