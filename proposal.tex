\documentclass[a4paper, 11pt]{article}
%\usepackage[parfill]{parskip}

\title{Some Title I haven't decided yet}
\author{Keimi Okamoto}


\setlength{\parindent}{0pt}
\begin{document}

\maketitle
%........................
% Contents
%........................
\tableofcontents
\clearpage

%........................
% Abstract
%........................
\abstract{Abstract}
blah blah
\clearpage

%........................
% Introduction
%........................
\section{Introduction}

\subsection{Background}
Over production of food is a global issue that has negative social, environmental and economical impacts with reports of 1.2 - 2 billion tonnes of waste produced by the world. The food industry is sustained by a complex multifaceted supply chain. Having such complex infrastructure results in ample opportunity for waste to occur. 

At the start of the chain exists the producers, agriculturalists and farmers who cultivate the plants and livestock. These entities serve as the primary source of supply to the market and respond to orders placed by the retailers. Over production is often encouraged by retailers to compensate for an unfruitful harvest or sudden and unexpected rise in demand. 

The intermediator between producer and consumer are the retailers and vendors, the most dominant being Supermarkets, namely, Sainsbury?s, Tesco, Asda and Morrison. Perpetual competition for marketshare lead to aggressive promoting and competitive pricing wars, with such high stakes involved quantity control becomes less of an focus. Ascetically unpleasing but perfectly consumable food is rejected and used as feed for livestock or discarded. Sales forecasts are also influenced by weather and events, errors in predictions can amount to wastage or the need for storage to sell at a later time. Having more products than necessary is unfavourable for business as the rental of warehouses are expensive and real-estate is valuable, idle stock can cost millions to house annually.

At the end of the chain are the consumers. Consumers are regularly influenced by enticing buy get one free offers and discounts. Retailers discount items that are nearing expiration and sold to the consumer as an attempt to compensate for a potential losses. This method of damage control, while beneficial to the supermarkets has negative financial implications on the consumers. Loss can also occur due to basic human errors of simply forgetting to consume purchased items in time. With many people having busy schedules it is difficult to keep track of past purchases and expiration dates, inadvertently contributing to the statistics.

Item tracking is already deployed in the grocery supply chain with the primary intent as inventory management so suppliers are able to track their assets as they move through the chain. Lack of transparency in the supply chain results.
\vspace{\baselineskip}

// Page 12: image add The language of food waste along the supply chain 


\subsection{Problem}
The main focal point of this project is at the consumer level for the grocery supply chain, for the reason that the largest amount of waste is being generated here. The UK alone has been estimated to produces 15million tonnes of food waste every year. The largest culprit is the domestic household amounting to 7 million tonnes of waste. Tightly couples so must be considered at times as a whole and other stages will be mentioned.

\paragraph{Environmental} When food is wasted, this is the direct repercussion of over production and a needless contribution to the expanding carbon foot print. Processes such as pesticide application, cooking, packaging creation and disposal, distribution and temperature controlled storage all require enormous amounts of fuel and energy. Not enough space is available to accommodate for waste, with the rate in which waste is produced it is not maintainable causing landfill sites to overflow.

\paragraph{Economical} Typically a UK household has been reported to throw away an average of \pounds940 worth of food annually. This amounts to roughly 50kg of food which puts pressure on the council for waste collection, management and recycling all which can result in raised taxes. 

\paragraph{Social}Influenced by the retailers and succumbing to the bargain deals, customers over purchase food and can cause over consumption. Needlessly consuming to avoid loss can cause health risks such as obesity, diabetes, high blood pressure, the source of cardiovascular diseases. The consumption of food can lead to the topic of the lack of transparency retailers offer when incomes to the origins of food. By law processed What are food is made up of? Laws here. Supply and demand fluctuation.

\paragraph{Human memory}
Consumers are reliant on memory to keep inventory of purchased products and foods usually have varied expiration dates that can cause difficulty in prioritisation of consumption. Human memory has been proven to be inaccurate

\paragraph{Lifestyle}Busy schedules dissuade people to use produces brought in advance and instead opt for the quick and easier choice of eating out, thus, items purchased with the intentions of consumption end up as waste. Fresh produces also usually vary in expiration dates and the synchronisation of dates and matching ingredient to create an appetising dish can be time-consuming. Households usually have multiple residents living and double purchasing of items is common due to lack of communication. 

\vspace{\baselineskip}
Image of pie goes here: put in Appendix Image will explain-Supermarkets and other retailers contribute almost 2 million tonnes to the statistics and Approximately 3.6 million tonnes of the waste comes from the manufactures such as farmers and the agricultural industry.


\subsection{Current measures to combat waste}

\subsection{Internet of things and SmartCities}
\paragraph{A bit of history} The first appliance to go online was a Coca-Cola vending machine developed in Carnegie Mellon University in 1982. Users were able to connect via the internet and check if the canned beverages were chilled, this information would be the deciding factor on whether the user would make the trip to the machine. This technological advancement gave an insight into a new era were objects were able to cater to our immediate needs, depending on the current circumstance and deliver us information with which we are able to make an informed decision.

\paragraph{SmartCities} From the success of the lead to?  the SmartCities are being erected in many cities such as blah, the idea behind the development is machine to machine communicating and the enablement of a once passive and inanimate object to be able to actively communicate with other ?things? through a network, sharing data and working harmoniously to maximise efficiency to permissively aid our everyday lives. 

Complex pipelines such as the food supply chain is a prime candidate for this type of technological advancement. With the ability and accessibility the internet offers could produce a transparent pipeline enhancing multiple as well as the miniaturisation of products. Minimising waste.

With the surge in popularity in smart devices in the recent decade, in particular the smartphone, which has been positively accepted and integrated into our lives has undoubtably accelerated in the movement of the ?Internet Of Things?. 

\paragraph{RFID vs Barcodes}
Barcodes are printed, therefor can not change.

\subsection{Importance of Item-level Identification}
\emph{'The amount of food waste in the industrialised countries exceeds the total first
production of the whole continent of Africa. This is an incredible waste of human effort
and environmental and economic cost. I say, ?On some estimates?, because we very
rapidly found that the estimates in this field are rather difficult, which limits the degree
to which the EU can play as effective a role as it perhaps ought. We found that
measurement of food waste at different stages of the chain and between different
countries was pretty incompatible. Until that is resolved, the EU level probably has to
be aspirational, exculpatory and a matter of learning from best practice. 12'}
\paragraph{Food Safety and traceability}
\paragraph{Transparency}

%........................
% Aims & Objectives
%........................
\section{Aims and Objectives}
\subsection{Aims}
Human memory is depended upon to keep track of our supply at home but repeatedly we simply forget and inadvertently contribute to the statistics. By applying a similar concept as the Coca-Cola machine, we can delegate the responsibility of inventory keeping to the fridge. The fridge is able to behave as a monitor and keep track of items and expiration dates residing inside it. Owners are alerted when the product is nearing the end of it?s life span, giving ample time to use it thus avoiding waste. Simply by putting the fridge online it becomes accessible from anywhere with an internet connection.

One of the core issues discussed in the previous chapter was the invasive advertisement of offers and promotions by retailers. It is easy to over purchase without truly knowing what you possess back at home, not even a carefully prepared shopping list is enough to dissuade the shopper from purchasing an extra bag of salad with the promise of a free one. Retailers have been exploiting this vulnerability for decades but by providing consumers with an option to consult their fridge before making the decision to purchase will keep the shopper honest and able to pre-emptively identify if the investment will amount to waste. The fridge can act as a safeguard for the shopper saving the household hundred of pounds annually and most importantly reducing the carbon footprint. 

Supermarkets strive to meet a demand in the market, often this is a contrived demand that is orchestrated by marketing and persuasion to boost revenue. The primary source of waste of resides with retailers but is pushed down the chain and dumped with the consumers. By capping demand this would create a ripple effect that would keep the waste at bay with the retailers that would discourage the over production of food, ultimately targeting the root cause of the problem.

Although this may seem negative for the supermarkets they will also be able to benefit from this technology as millions in rent is waster by idle stock being housed in warehouses, this is a clear sign of waster resources and revenue. 
\subsection{Objectives}
\clearpage


%........................
% Development
%........................
\section{Development}
\subsection{Methodology}
\subsection{Framework}
\subsection{Architecture}
\subsection{Tools}

\clearpage

%........................
% Conclusion
%........................
\section{Conclusion}
\subsection{Limitations}

\clearpage


%........................
% Schedule
%........................
\section{Schedule}
\subsection{Timetable}

\clearpage


%........................
% Bibliography
%........................
\begin{thebibliography}{11}

\bibitem{code}
	fjdsfjds

\end{thebibliography}

\end{document}
