\documentclass[a4paper, 11pt]{article}
%\usepackage[parfill]{parskip}

%\title{Some Title I haven't decided yet}
%\author{Keimi Okamoto}
\usepackage[T1]{fontenc}

\newlength{\drop}

%%%%%%%START%%%%%%%
\begin{document}
  \begin{titlepage}
    \drop=0.1\textheight
    \centering
    \vspace*{\baselineskip}
    \rule{\textwidth}{1.6pt}\vspace*{-\baselineskip}\vspace*{2pt}
    \rule{\textwidth}{0.4pt}\\[\baselineskip]
    {\Large{MSc Computer Science\\[0.3\baselineskip] }}
    {\huge{Project Proposal\\[0.3\baselineskip] }}
	
    \rule{\textwidth}{0.4pt}\vspace*{-\baselineskip}\vspace{3.2pt}
    \rule{\textwidth}{1.6pt}
    \\[\baselineskip]
    \scshape
    {\Large Some Title That I haven?t Decided Yet \\}
    Location, date from--to\par
    \vspace*{2\baselineskip}
    %Edited by \\[\baselineskip]
    {\normalsize\emph{Supervisor: }{\large Professor George Roussos\par}}
    {\normalsize\emph{Author: }{\large Keimi Okamoto\par}}
    
    {\itshape 2015}
    \vfill
    {\large BIRKBECK UNIVERSITY OF LONDON\par}
{\footnotesize DEPARTMENT OF COMPUTER SCIENCE \& INFORMATION SYSTEMS}\par
  \end{titlepage}
  
%\maketitle
%........................
% Contents
%........................
\tableofcontents
\clearpage

%........................
% Abstract
%........................
\abstract{Abstract}
blah blah
\clearpage

%........................
% Introduction
%........................
\section{Introduction}

\subsection{Background}
The over production of food is a global issue with waste generated by the world exceeding two billion tonnes annually. This has disastrous social, environmental and economical consequences. 

The food industry is sustained by a multifaceted supply chain, such complex infrastructure gives ample opportunity for waste to occur. A clear understanding of the chain is necessary to identify the vulnerable components and prevent waste.

\paragraph{Producers}At the start of the chain exists the producers, agriculturalists and farmers who cultivate the plants and livestock. These entities serve as the primary source of supply to the market and respond to orders placed by the retailers. Over production is often encouraged to compensate for an unfruitful harvest or sudden and unexpected rise in demand. 

\paragraph{Retailers}The intermediator between producer and consumer are the retailers and vendors, the most dominant being Supermarkets, namely, Sainsbury?s, Tesco, Asda and Morrison. Perpetual competition for marketshare leads to aggressive promoting and competitive pricing wars, unfortunately with millions of pounds at stake quantity control becomes less of an priority. Ascetically unpleasing but perfectly consumable food is rejected and discarded. Errors in sales forecasts can amount to wastage or lead to idle stock occupying valuable real-estate. Having more products than necessary is unfavourable for business as the rental of warehouses can cost millions annually.

\paragraph{Consumers}At the end of the chain are the consumers. Consumers are regularly influenced by enticing ?buy-one-get-one-free? offers and other multi-buy promotions. Retailers markdown items that are nearing expiration and sell it to the consumer as an attempt to compensate for potential losses. This method of damage control while beneficial to the supermarkets, has negative financial implications on the consumers. Loss can also occur due to basic human errors of simply forgetting to consume purchased items in time. With many people having busy schedules it is difficult to keep track of past purchases and expiration dates individuals consequently inadvertently contributing to the statistics. 

\vspace{\baselineskip}

The largest culprit of food waste, and the focal point of this project are the consumers. The United Kingdom alone is estimated to generate15million tonnes of food waste every year, 7 million of which is accountable to domestic households. Although consumers alone can not be held responsible. Supermarkets strive to meet a demand in the market, often this is a contrived demand that is orchestrated by marketing and persuasion to boost revenue. The primary source of waste residing with retailers is pushed down the chain and ultimately dumped with the consumers. By capping demand this would create a ripple effect that would keep the waste at bay with the retailers that would discourage the over production of food and targeting the root cause of the problem.
This project will take a consumer-centric approach to the issue of waste. Aiming to explore the interoperability of objects at the consumer level to maximise resources and thus, reducing food waste. Utilising radio frequency identification (RFID) at the item-level identification of products is achieved to identify potential waste and in addition, promulgate a transparent supply-chain.

\vspace{\baselineskip}
\vspace{\baselineskip}
\vspace{\baselineskip}

\subsection{Problem}
The largest culprit of food waste and the focal point of this project are the consumers. The United Kingdom alone is estimated to generate15million tonnes of food waste every year, 7 million of which is accountable to domestic households. Below are the issues surrounding this growing epidemic. 

\paragraph{Environmental}When food is wasted, this is the direct repercussion of over production and a needless contribution to the expanding carbon foot print. Processes such as pesticide application, cooking, packaging creation and disposal, distribution and temperature controlled storage all require copious amounts of fuel and energy. Waste is being generated at such an alarming rate maintaining this in landfill sites is becoming less feasible.

\paragraph{Economical}Typically a UK household has been reported to throw away an average of \pounds940 worth of food annually. This amounts to roughly 50kg of waste that must be collected, managed and recycled putting pressure on the council and all which can result in higher taxes and wasted resources.

\paragraph{Social} Influenced by the retailers and succumbing to the bargain deals, customers frequently over purchase food and can cause over consumption. Needlessly consuming to avoid loss can cause health risks such as obesity, diabetes, high blood pressure, cardiovascular diseases that are life threatening and can impact life expectancy rates. 

\paragraph{Human Memory} Failed stock keeping of household items is one of the primary causes of food expiring before consumption. The human brain is constrained by the limited capacity to store and recall information. Research by George Armitage Miller, a prominent figure in the field of cognitive phycology discovered that the number of objects an average human can hold in working memory is seven, give or take two.\cite{2} Foods vary in categories such as meats, fish, fruit, vegetables and nearly all come with different expiration dates, relying on memory alone to recall is unfeasible. 

\paragraph{Lifestyle} Busy schedules dissuade people to use produces brought in advance and instead opt for the quick and easier choice of eating out, thus items purchased with the intentions of consumption end up as waste. Fresh produces also usually vary in expiration dates and the synchronisation of dates and combining compatible ingredient to create an appetising dish can be time-consuming and an argues task. Households usually have multiple residents living and double purchasing of items is common due to lack of communication. 

\vspace{\baselineskip}
\vspace{\baselineskip}
\vspace{\baselineskip}

\subsection{Current Measures to Combat Waste}

\paragraph{Governments} Various campaigns have been launched with the purpose to educate people on the implications of food waste and waste prevention methods. Such organisations as Waste \& Resources Action Programme (WRAP), a registered charity part funded by the UK Government have been raising awareness by interacting with communities and working to promote waste avoidance. A campaign launched by WARP ?Love Food, Hate Waste? (LFHW), which primarily operates through an interactive website advocating recycling and provides the viewer with helpful tips. LFHW utilise social media tools such as Twitter and Facebook to gain recognition and publicity. WRAP have also released an mobile application with features such as shopping lists, recipe suggestions, portion size suggestion in accordance to the number of people serving. A flaw in this logic is that produces are often prepackaged and the weight is fixed, even if the consumer carefully weighed out the correct portion as suggested there is still leftover produce that can end up as waste, thus not addressing the problem of waste management. Another example mobile application is ?Smart Cooking?, developed by the Netherlands Nutrition Centre Foundation (NNCF), aimed to educate people on the reduction of food waste and includes similar features to LFHW.

\paragraph{Wageningen University}Researchers in the Netherlands are working with the Netherlands Packaging Centre who offer packaging solutions throughout the supply chain, and with NXP Semiconductors have developed a sensor enabled radio frequency identification (RFID) tag developed to monitor environmental changes in the supply chain. The Pasteur sensor tag has the capacity of measuring environmental conditions such shifts in temperature and gas conditions registered during transportation and various stages of storage. This data is then calculated to give an accurate reading of the shelf life of a product. At the currents state this is only available at the level of producer to supplier level for large crates of produces but in future item level tagging being considered. 

\paragraph{Pecking University}Scientists in Beijing have also developed a colour coded smart tag that uses nanotechnology to indicate when food is spoiling. The metallic nanorods in the gel mimic the length of time microbes propagate in foods, the more bacteria the further it is in the decomposition process. The tag can also react to varying tempters that can have an effect on the shelf life of a product, this could potentially remove the necessity of sell-by-dates. The accuracy of this technology is currently being tested. Meticulous testing is needed when dealing with such sensitive materials as food to avoid any inaccuracy that could pose a potential health risk.

\paragraph{Smart Fridges} Interactive appliances for home food management are designed for integration with the Smart Home and automation of food ordering. The user will scan products using a installed barcode reader, the fridge is able to keep track of foods and suggest a dish depending on the items residing inside it. It can switch on the cooker to the desired temperature setting and timer using wireless connections and provide instructions on how to prepare the dish. The intention was for it to analyse the consumers purchasing habits and to pre-emptively place the orders so the consumer always has a stocked fridge. 

\paragraph{TooSkee \& LeanPath} Software food management applications focusing on reducing waste and using smartphones to keep track of purchased items. Receipts are scanned and items are logged. Much like the other mobile applications it will suggest dishes and has an additional feature of reminding the user to consume products before the expiration date.		

\vspace{\baselineskip}
While manual campaigning and promoting can be effective and inspirational the labour force required to generate and sustain interest is costly and impractical. The smart fridge that was heavily anticipated was somewhat anti-climatic as flaws in the practicality of the product became evident. Items had to be manually entered, consumers were tied to a single retailer and the appliance was very expensive, costing over \$20,000. Mobile phone applications are the most accessible but users are extremely unforgiving of a poor interface design or performance such as slow content loading. With so many apps available with a tap of a screen, apps have become disposable. 

Cutting edge innovations such as the atmosphere and temperature sensing RFID tag and the nanotechnology tag are undeniably beneficial to the supply chain. The only bottleneck for the RFID  tag is the cost of production at the individual item level. Nanotechnology could potentially replace printed use-by-dates on products  and could provide users with the an accurate reading of the longevity of food but this requires the user to manually open the fridge and memorise the colour of the tags. Presently there is no method for the tags to communicate, the data generated could be valuable for waste management and even retailers. Transparency throughout the supply chain is paramount at this moment in time and is achievable with the aid of machine to machine communication. Products and prototypes mentioned here can be enhanced and with the necessary infrastructure to support it. This leads on to the next topic of the movement of the Internet Of Things.

\vspace{\baselineskip}
\vspace{\baselineskip}
\vspace{\baselineskip}


\subsection{Internet of Things and SmartCities}
\paragraph{A bit of history} The first appliance to go online was a Coca-Cola vending machine developed in Carnegie Mellon University in 1982. Users were able to connect via the internet and check if the canned beverages were chilled, this information would be the deciding factor on whether the user would make the trip to the machine. This technological advancement gave an insight into a new era were objects were able to cater to our immediate needs, depending on the current circumstance and deliver us information with which we are able to make an informed decision. With the enablement of machine to machine communications, a once passive and inanimate object is able to actively communicate with other ?things? through a network, sharing data and working harmoniously to maximise efficiency to permissively aid our everyday lives.

\paragraph{SmartCities} The motivation behind Smart Cities is the idea of a self sustaining ecosystem made up of active object, supporting the efficient and economical utilisation of resources. The digital regulation of sectors such as, mobility, home, energy and waste management assists a better quality of life for its citizens. Information and communications systems are designed to deal with autonomous fault detection and self-healing intelligence with minimal, if not, the complete elimination of human intervention. 

This notion of efficiency driven, interconnected networks can be applied to the supply chain. Interleaving processes can be machine managed, making it less error prone and complex pipelines and components become transparent. Allowing the promulgation of waste occurrence to it?s counterparts so that preemptive waste prevention measures could be established in advance. An example of this is if a store shelf holding boxes of cornflakes can report to the start of the chain that there are still x-units to sell, so for the next x dates cornflakes need not be produced. Giving opportunity for demand to catch up and reducing back-log of stock.

\paragraph{Product identification: RFID vs Barcodes}Product identification and registration is fundamental for retailers to monitor stock and analyse sales totals. To accomplish this task, barcodes have been implemented in the supply chain since the 1970?s. This method of identification accelerate and digitalised the checkout process. 

For the barcode to be read successfully there must be no obstructions between the laser and the barcode, this includes dirt or scratches that distort the image. The laser must be kept parallel to the barcode for a successful read and simultaneous scanning is not possible. It is also note worthy to mention that barcodes are unique to the product type but not at the item-level. For example in a crate filled with milk made by the same manufacture will all have the same barcode thus a machine would not be able to distinguish one item from another. However they are universally recognised and inexpensive to print. 

Unlike barcodes lasers are not needed to for reads. RFID utilises electromagnetic radio fields for communication, so long long as the tag is within the vicinity of the field it is able be read or written and even allow multiple tags to be read simultaneously. Tag have varied memory capacity but typically very low. Usually the size of a URL can be written, this is enough to bridge between object and the web where additional data can be stored or updated as external data storage is abundant. URL?s also provide individuality to an object, allowing customised granular information to be stored and accessible at the item-level.


\vspace{\baselineskip}
\vspace{\baselineskip}
\vspace{\baselineskip}

\subsection{Importance of Item-level Identification}
\emph{'The amount of food waste in the industrialised countries exceeds the total first
production of the whole continent of Africa. This is an incredible waste of human effort
and environmental and economic cost. I say, ?On some estimates?, because we very
rapidly found that the estimates in this field are rather difficult, which limits the degree
to which the EU can play as effective a role as it perhaps ought. We found that
measurement of food waste at different stages of the chain and between different
countries was pretty incompatible. Until that is resolved, the EU level probably has to
be aspirational, exculpatory and a matter of learning from best practice. 12'}

But item-level tagging can benefit all areas of the supply chain, there is even a movement, Electronic product code (EPC)Developed in Massachusetts Institute of Technology Auto-ID Centre, with the aim of providing a unique identification for every physical object in the world. EPC Network

\paragraph{Transparency \& Consumer Rights}
Regulations of the EU parliament and council it has been stated that abattoirs and agriculturalists must provide documentation of the source of their produce(cite) to retailers and officials but is not enforced at the consumer level. By law(cite) producers must state on the label the country of origin of the meat but does not apply to processed meats, such as hamburgers and pies. (cite article 18) Providing that the animal was slaughtered in the same country, different batches of meat may also be mixed. (In article 18 by the) Consumers are often oblivious to what their food is made of. 

As livestock are already tagged with RFID tags, if the cattle holding information such as the environment they were brought up in, the feed they consumed and drugs they were administered could all be provided to assist the buyer to make a better decision. Such areas such as japan where radiation is present and health concerns are a high priority the need for transparency regarding food is evident.  

\paragraph{Amendment of human errors}
Information such as allergy and nutrition that is printed can be digitalised and accessible online. Labels not having the correct information so recalled and wasted. If this was using RFID it could remotely be updated issuing consumers of the error and returned. Not that it was unfit for consumption but just coz it broke the law in it. what a waste yo.


\paragraph{Food Safety and traceability}
In the past there has been numerous incidents where products have been recalled due to the presence of bacteria or abnormalities in foods. A notable incident is the 2013 meat adulteration scandal in the EU where traces of horse meat where discovered in foods such as minced meat and ready prepared meals. The time and resources to trace trough the supply chain was estimated to have cost the Food Standard Agency (FSA) \pounds900,000 between 2011?2012. And a further \pounds1.6 million between 2012?13 \cite{3}. Other casualties include the blameless producers who?s reputations and integrity were disputed due to inaccurate data that implicated them as the guilty. With item-level identification the contaminated produce could be traced back immediately and the products recalled. With the use of RFID if one animal is used in various products, thighs for roasting meats, bones for, ofel in mince and this animal was the contaminated one all items with that particular code can be instantly tracable, risk of illness compartmentalised and damage controlled. 
By having itemised identification and recalls are issued indiscriminately this amounts to waste and cost for the investigation.
(*1)Environment, Food and Rural Affairs Committee: Evidence,
example:Article 3 of Commission Implementing Regulation EU

\paragraph{Supply Chain Intelligence} As well as waste management Item-level identification can benefits to other areas of the supply chain such as behavioural analysis of consumers
security concerns 
issue of security has to be able to opt in or out

The bottleneck for this technology is 

\clearpage
%........................
% Aims & Objectives
%........................
\section{Aims and Objectives}
\subsection{Aims}
Human memory is depended upon to keep track of our supply at home but repeatedly we simply forget and inadvertently contribute to the statistics. By applying a similar concept as the Coca-Cola machine, we can delegate the responsibility of inventory keeping to the fridge. The fridge is able to behave as a monitor and keep track of items and expiration dates residing inside it. Owners are alerted when the product is nearing the end of it?s life span, giving ample time to use it thus avoiding waste. Simply by putting the fridge online it becomes accessible from anywhere with an internet connection.

One of the core issues discussed in the previous chapter was the invasive advertisement of offers and promotions by retailers. It is easy to over purchase without truly knowing what you possess back at home, not even a carefully prepared shopping list is enough to dissuade the shopper from purchasing an extra bag of salad with the promise of a free one. Retailers have been exploiting this vulnerability for decades but by providing consumers with an option to consult their fridge before making the decision to purchase will keep the shopper honest and able to pre-emptively identify if the investment will amount to waste. The fridge can act as a safeguard for the shopper saving the household hundred of pounds annually and most importantly reducing the carbon footprint. 

Supermarkets strive to meet a demand in the market, often this is a contrived demand that is orchestrated by marketing and persuasion to boost revenue. The primary source of waste of resides with retailers but is pushed down the chain and dumped with the consumers. By capping demand this would create a ripple effect that would keep the waste at bay with the retailers that would discourage the over production of food, ultimately targeting the root cause of the problem.

Although this may seem negative for the supermarkets they will also be able to benefit from this technology as millions in rent is waster by idle stock being housed in warehouses, this is a clear sign of waster resources and revenue. 
\subsection{Objectives}
\subsection{Limitations}
\clearpage


%........................
% Development
%........................
\section{Development}
\subsection{Methodology}
\subsection{Framework}
\subsection{Architecture}
\subsection{Tools}

\clearpage

%........................
% Schedule
%........................
\section{Schedule}
\subsection{Timetable}

\clearpage

%........................
% Bibliography
%........................
\begin{thebibliography}{11}

\bibitem{code}
	fjdsfjds
	
\bibitem{2}
	G.A Miller, \emph{Magical Number Seven, Plus or Minus Two: Some Limits on Our Capacity for Processing Information}, vol. 63. Cambridge, MA: The Psychological Review, 1956.

\bibitem{3}
Environment, Food and Rural Affairs Committee: Evidence,
example:Article 3 of Commission Implementing Regulation EU


\end{thebibliography}

\end{document}

%%%
\setlength{\parindent}{0pt}
\begin{document}


\end{document}
